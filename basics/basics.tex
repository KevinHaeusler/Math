\section*{Allgemein}
\begin{multicols}{3}
\subsection*{Wachstum und Verfall}
\textbf{Wachtsumsfaktor:}
\begin{equation*}
    q = 100\ \% + p\ \% = 1 + \frac{p}{100}
\end{equation*} 
\textbf{Verdoppelungszeit:}
\begin{equation*}
    t_V = \frac{\ln(2)}{\ln(q)}
\end{equation*}
\textbf{Abnahme:}
\begin{equation*}
    B(t) = {\color{red}m} \cdot t + b \quad \text{mit } {\color{red}m < 0}
\end{equation*}

\subsection*{Summe und Produkte}
\textbf{Summezeichen:} \\
Es sei:  n, k $\in$ Z und n $\geq$ k
\begin{equation*}
    \sum_{k=1}^{n} a_k = a_1 + a_2 + a_3 + \ldots + a_n 
\end{equation*}
\textbf{Produktzeichen:}
\begin{equation*}
    \prod_{k=1}^{n} a_k = a_1 \cdot a_2 \cdot a_3 \cdot \ldots \cdot a_n
\end{equation*}
k = Laufvariable, Laufindex  \\
1 = Startwert  n = Endwert\\
$a_{k}$ ist die Funktion bezueglich der Laufvariable \\

\subsection*{Aussagen, Logik, Mengen}
\begin{tabularx}{\columnwidth} {
        | >{\raggedright\arraybackslash}c
        | >{\raggedright\arraybackslash}X |}
    \hline
    \textbf{*}            & \textbf{Bedeutung}                                           \\ \hline
    $\varnothing$ oder {} & Leere Menge                        \\ \hline
    $x\in A$              & Element x ist in Menge A                          \\ \hline
    $x\notin A$           & Element x ist nicht in Menge A                    \\ \hline
    $A\subset B$          & A ist eine Teilmenge von B                                   \\ \hline
    $A\cap B$             & Schnittmenge von A und B                                     \\ \hline
    $A\cup B$             & Vereinigunsgsmenge von A und B                               \\ \hline
    $A\backslash B$       & Differenzbildung, Menge A ohne B                         \\\hline
    $\bar{A}_B$           & $:= \{x \,|\, x \in B \enspace \wedge \enspace x \notin A\}$ \\\hline
\end{tabularx}


\def\firstcircle{(0,0) circle (1cm)}
    \def\secondcircle{(0:1.2cm) circle (1cm)}

    \colorlet{circle edge}{blue!50}
    \colorlet{circle area}{blue!20}

    \tikzset{filled/.style={fill=circle area, draw=circle edge, thick},
        outline/.style={draw=circle edge, thick}}

    \setlength{\parskip}{5mm}
    % Set A and B
    \begin{tikzpicture}
        \begin{scope}
            \clip \firstcircle;
            \fill[filled] \secondcircle;
        \end{scope}
        \draw[outline] \firstcircle node {$A$};
        \draw[outline] \secondcircle node {$B$};
        \node[anchor=south] at (current bounding box.north) {$A \cap B$};
    \end{tikzpicture}
    \begin{tikzpicture}
        \draw[filled] \firstcircle node {$A$}
        \secondcircle node {$B$};
        \node[anchor=south] at (current bounding box.north) {$A \cup B$};
    \end{tikzpicture}

    % Set A but not B
    \begin{tikzpicture}
        \begin{scope}
            \clip \firstcircle;
            \draw[filled, even odd rule] \firstcircle node {$A$}
            \secondcircle;
        \end{scope}
        \draw[outline] \firstcircle
        \secondcircle node {$B$};
        \node[anchor=south] at (current bounding box.north) {$A\backslash  B$};
    \end{tikzpicture}
    \begin{tikzpicture}
        \draw[filled, even odd rule] \firstcircle node {$A$}
        \secondcircle node{$B$};
        \node[anchor=south] at (current bounding box.north) {$\overline{A \cap B}$};
    \end{tikzpicture}\\
    \textbf{Aussagenlogik:}\\
    Eine Aussage beschreibt einen Sachverhalt (durch Worte oder Symbole), der
    entweder wahr oder falsch ist.\\

    \scriptsize
    \begin{tabular}{@{ }c@{ }@{ }c | c@{ }@{ }c@{ }@{ }c@{ }@{ }c@{ }@{ }c | c@{ }@{ }c@{ }@{ }c@{ }@{ }c@{ }@{ }c | c@{ }@{ }c | c@{ }@{ }c@{ }@{ }c@{ }@{ }c@{ }@{ }c@{ }@{ }c}
        A & B &  & A & $\land$            & B &  &  & A & $\lor$             & B &  & $\lnot$            & B &  & A & $\lor$             & $\lnot$ & B & \\
        \hline
        T & T &  & T & \textcolor{red}{T} & T &  &  & T & \textcolor{red}{T} & T &  & \textcolor{red}{F} & T &  & T & \textcolor{red}{T} & F       & T & \\
        T & F &  & T & \textcolor{red}{F} & F &  &  & T & \textcolor{red}{T} & F &  & \textcolor{red}{T} & F &  & T & \textcolor{red}{T} & T       & F & \\
        F & T &  & F & \textcolor{red}{F} & T &  &  & F & \textcolor{red}{T} & T &  & \textcolor{red}{F} & T &  & F & \textcolor{red}{F} & F       & T & \\
        F & F &  & F & \textcolor{red}{F} & F &  &  & F & \textcolor{red}{F} & F &  & \textcolor{red}{T} & F &  & F & \textcolor{red}{T} & T       & F & \\
    \end{tabular}
    \normalsize

    \begin{tabularx}{\columnwidth} {
        | >{\raggedright\arraybackslash}c
        | >{\raggedright\arraybackslash}X
        | >{\raggedright\arraybackslash}c|}
    \hline
    \textbf{*}            & \textbf{Bedeutung}                        & \textbf{Beispiel}             \\ \hline
    $|A|$                 & Kardinalität/Mächtigkeit      & A = {1;2}                     \\
                          & Anzahl Elemente               & |$|A|$ = 2                    \\ \hline
    $\land$               & Konkuktion/UND A $\land$    & A $\land$ B                   \\ \hline
    $\lor$                & Disjunktion/ODER A $\lor$ B  = Wahr wenn  & {-1;0;1;}                     \\\hline
    $\neg$                & Negation A = W $\neg$A = F      & $\neg $A                      \\\hline
    $\implies$            & Implikation: Daraus folgt                 &                               \\ \hline
    $\Longleftrightarrow$ & äquivalenz  &                               \\ \hline
    $\forall$             & für Alle                                  & $\forall$ $x \in \mathbb{N}$  \\ \hline
    $\exists$             & Es Existiert                              & $\exists $ $x \in \mathbb{N}$ \\ \hline
    \end{tabularx}
\subsection*{Misc}

\newpage
\end{multicols}