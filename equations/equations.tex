\section*{Gleichungen}
\begin{multicols*}{3}
    \subsection*{Lineare Gleichung}
    \textbf{Definiton} \\
    Eine Gleichung, die sich in die Form $ax + b = 0$ bringen lässt, heisst lineare Gleichung. Wir können lineare Gleichungen daran erkennen, dass die Variable nur in der 1. Potenz auftritt, also kein $x^2$, $x^3$\dots enthalten. \\\\
    \textbf{Lösen einer Linearen Gleichung}
    \begin{enumerate}
        \item Gleichung nach $x$ auflösen
        \item Lösungsmenge aufschreiben
    \end{enumerate}
    \subsection*{Quadratische Gleichungen}
    \textbf{Definiton} \\
    Gleichungen, die sich auf die Form $ax^2 + bx + c = 0 \quad (a, b, c \in \mathbb{R}; a \neq 0)$
    bringen lassen, heissen quadratische Gleichungen. 
    \\~\\
    Wir können quadratische Gleichungen daran erkennen, dass die Variable $x$ in der 2. Potenz $x^2$, aber in keiner höheren Potenz vorkommt. 
    \\~\\
    Es gibt 4 Arten/Formen von Quadratischen Gleichungen.
    \begin{enumerate}
        \item $ax^2 + bx + c = 0 \quad (a, b, c \in \mathbb{R}; a \neq 0)$
        \item $ax^2 + bx = 0 \quad (a, b \in \mathbb{R}; a \neq 0)$
        \item $ax^2 + c = 0 \quad (a, c \in \mathbb{R}; a \neq 0)$
        \item $ax^2 = 0 \quad (a \in \mathbb{R}; a \neq 0)$
    \end{enumerate}
    \textbf{Lösung einer Reinquadratische Gleichung $ax^2 = 0$}\\~\\
    Reinquadratische Gleichungen ohne Absolutglied besitzen als einzige Lösung die Null.\\~\\
    \textbf{Lösung einer Reinquadratische Gleichung mit Absolutglied $ax^2 + c = 0$}

    \begin{enumerate}
        \item Gleichung nach $x^2$ auflösen
        \item Wurzel ziehen
        \item Lösungsmenge aufschreiben
    \end{enumerate}
    \textbf{Lösung einer Gemischtquadratische Gleichungen ohne Absolutglied $ax^2 + bx = 0$}

    \begin{enumerate}
        \item Quadratische Gleichung in Normalform bringen
        \item $x$ ausklammern
        \item  Faktoren gleich Null setzen
        \item Gleichung nach $x^2$ auflösen
        \item Lösungsmenge aufschreiben
    \end{enumerate}

    \subsection*{Mitternachtsformel}
    Gemischtquadratische Gleichungen $ax^2 + bx + c = 0$ mit Absolutglied lösen wir mit der Mitternachtsformel:
    \[x_{1/2} = \frac{-b \pm \sqrt{b^2 - 4ac}}{2a}\]
    \textbf{Fallunterscheidung:} \\
    \begin{align*}
        x_{1} &= \dfrac{-b - \sqrt{b^2 - 4ac}}{2a} \\
        x_{2} &= \dfrac{-b + \sqrt{b^2 - 4ac}}{2a}
    \end{align*}
    \textbf{Übersicht}\\~\\
    \begin{tabularx}{1.15\columnwidth} {
            | >{\raggedright\arraybackslash}X
            | >{\raggedright\arraybackslash}X
            | >{\raggedright\arraybackslash}X |}
        \hline
        \textbf{}                               & \textbf{Allgemeine Form}                       & \textbf{Normalform}                           \\ \hline
        Reinquadratisch ohne Absolutglied       & $2x^2 = 0$, $a = 2$, $b = 0$ und $c = 0$       & $x^2 = 0$, $a = 1$, $b = 0$ und $c = 0$       \\\hline
        Reinquadratisch mit Absolutglied        & $2x^2 -8 = 0$, $a = 2$, $b = 0$ und $c = -8$   & $x^2-4 = 0$, $a = 1$, $b = 0$ und $c = -4$    \\ \hline
        Gemischtquadrat- isch ohne Absolutglied & $2x^2-8x = 0$, $a = 2$, $b = -8$ und $c = 0$   & $x^2 -4x= 0$,  $a = 1$, $b = -4$ und $c = 0$  \\ \hline
        Gemischtquadrat- isch mit Absolutglied  & $2x^2-8x+6 = 0$, $a = 2$, $b = -8$ und $c = 6$ & $x^2-4x+3 = 0$, $a = 1$, $b = -4$ und $c = 3$ \\ \hline
    \end{tabularx}\\~\\
    \textbf{Regeln}\\~\\
    Wenn das lineare Glied fehlt, gilt b = 0. \\
    Wenn das absolute Glied fehlt, gilt c = 0. \\
    Wenn das $x^2$ allein steht, gilt a = 0 (wegen $1 \cdot x^2 = x^2$).\\
    Wenn das x allein steht, gilt b - 1(wegen $1 \cdot x = x$). \newpage
    \textbf{Lösen einer Quadratischen Gleichung mit Mitternachtsformel}
    \begin{enumerate}
        \item Gleichung in allgemeine Form bringen
        \item a,b,c aus der allgemeinen Form herauslesen
        \item a,b,c in die Mitternachtsformel einsetzen
        \item Lösung berechnen
        \item Lösungsmenge aufschreiben
    \end{enumerate}
    \subsection*{Bruchgleichungen}
    Wenn die Zähler der Brueche nur aus Zahlen bestehen, kann eine Kehrwertbildung sinnvoll sein.
    Den Kehrwert eines Bruchs erhält man durch Vertauschen von Zähler und Nenner.\\~\\
    \[\frac{{\colorbox{yellow}{$1$}}}{{\colorbox{orange}{$x$}}} = \frac{{\colorbox{yellow}{$2$}}}{{\colorbox{orange}{$x+1$}}} \Rightarrow  \frac{{\colorbox{orange}{$x$}}}{{\colorbox{yellow}{$1$}}} = \frac{{\colorbox{orange}{$x+1$}}}{{\colorbox{yellow}{$2$}}}\]
    \subsubsection*{Lösen einer Bruchgleichung}
    \begin{enumerate}
        \item Definitionsmenge bestimmen
        \item Gleichung nach $x$ auflösen
        \item Prïfen, ob der x-Wert in der Definitionsmenge ist
        \item Lösungsmenge aufschreiben
    \end{enumerate}
    \subsubsection*{Kehrwert}
    Wenn die Zähler der Brueche nur aus Zahlen bestehen, kann eine Kehrwertbildung sinnvoll sein.
    Den Kehrwert eines Bruchs erhält man durch Vertauschen von Zähler und Nenner.\\~\\
    \[\frac{{\colorbox{yellow}{$1$}}}{{\colorbox{orange}{$x$}}} = \frac{{\colorbox{yellow}{$2$}}}{{\colorbox{orange}{$x+1$}}} \Rightarrow  \frac{{\colorbox{orange}{$x$}}}{{\colorbox{yellow}{$1$}}} = \frac{{\colorbox{orange}{$x+1$}}}{{\colorbox{yellow}{$2$}}}\]


    \subsubsection*{Multiplikation übers Kreuz}
    Wenn auf beiden Seiten der Gleichung jeweils ein Bruch steht, kann eine Multiplikation ueber Kreuz sinnvoll sein.
    \[\frac{{\colorbox{yellow}{$1$}}}{{\colorbox{orange}{$x$}}} = \frac{{\colorbox{orange}{$2$}}}{{\colorbox{yellow}{$x+1$}}} \Rightarrow {\colorbox{yellow}{$1$}} \cdot {\colorbox{yellow}{$x+1$}} = {\colorbox{orange}{$2$}} \cdot {\colorbox{orange}{$x$}}\]

% AAAAAAAAAAAAAAA

\end{multicols*}
